\begin{resumo}[Abstract]
 \begin{otherlanguage*}{english}
  The use of software systems has become increasingly popular in modern society, following this growth rate
    , new learning methods and techniques emerge that use these systems as a way to improve teaching and learning. This job
    presents the step by step carried out in the construction of a gamified \ textit {web} tool, whose objective is to serve as a means
    to support the teaching and learning of programming in the discipline "Algorithms and Computer Programming", offered at the Gama (FGA) campus of
    University of Brasilia (UnB). The tool documented in this work, was built based on \ textit {framework octalysis} created by Yukai Chou. The pedagogical content covered by the tool is also presented
    as well as the process of choosing gamification techniques, technologies used, software architecture, data model and user experience.
   \vspace{\onelineskip}
 
   \noindent 
   \textbf{Key-words}: Gamification, Tool, Requirements, Development, Motivation, Learning, Programming, Architecture, Techniques.
 \end{otherlanguage*}
\end{resumo}
